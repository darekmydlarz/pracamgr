\documentclass{beamer}
\usetheme{Warsaw}
\usepackage[polish]{babel}
\usepackage[utf8]{inputenc}
\usepackage[T1]{fontenc}
\usepackage{amsmath}
\usepackage{graphicx}
\usepackage{tikz} % custom positions
\beamertemplatenavigationsymbolsempty % disable navigatoin bar
\usepackage{changepage} % adjustwidth


\def\tikzoverlay{%
   \tikz[baseline,overlay]\node[every overlay node]
}%




\title[Analiza sentymentu i geolokacja w sieciach społecznych]
{Możliwości powiązania 
\\ danych geolokacyjnych i analizy sentymentu \\
w analizie zachowań użytkowników \\ 
w wybranych portalach społecznościowych}
\author{Dariusz Mydlarz}

\institute[AGH Kraków]{
Promotor: dr inż. Anna Zygmunt
\\ \vspace{0.3cm}
Akademia Górniczo-Hutnicza im. Stanisława Staszica w Krakowie\\
Wydział Informatyki, Elektroniki i Telekomunikacji -- Katedra
Informatyki}

\date{Kraków, 10 grudnia 2014 roku}

\defbeamertemplate*{footline}{shadow theme}
{%
  \leavevmode%
  \hbox{\begin{beamercolorbox}[wd=.5\paperwidth,ht=2.5ex,dp=1.125ex,leftskip=.1cm plus1fil,rightskip=.1cm]{author in head/foot}%
    \usebeamerfont{title in head/foot}\insertshorttitle%
  \end{beamercolorbox}%
  \begin{beamercolorbox}[wd=.5\paperwidth,ht=2.5ex,dp=1.125ex,leftskip=.1cm,rightskip=.1cm plus1fil]{title in head/foot}%
    \usebeamerfont{author in
    head/foot}\insertshortauthor
    \hfill
    \insertframenumber/\inserttotalframenumber \end{beamercolorbox}}%
  \vskip0pt% 
}


% ==================================================================================
\begin {document}
% ==================================================================================
{
\setbeamertemplate{headline}{}
\setbeamertemplate{footline}{}
\begin{frame}
\maketitle
\end{frame}
}


% ==================================================================================
\begin{frame}{Agenda}
\tableofcontents
\end{frame}

% ==================================================================================
\section{Motywacja i cele}
% ==================================================================================
\begin{frame}{Motywacja i cele}
Cel -- zbadanie możliwości połączenia trzech dziedzin:
\begin{itemize}
\item analizy sieci społecznościowych
\item analizy sentymentu
\item geolokacji
\end{itemize}
\vspace{0.5cm}
Motywacje:
\begin{itemize}
\item dziedziny nie łączone wcześniej  
\item ogromna popularność serwisów społecznościowych
\item wykorzystanie szerokiej wiedzy zdobytej na studiach
\end{itemize}
\end{frame}

% ==================================================================================
\section{Sposób realizacji i przebieg prac}
% ==================================================================================
\begin{frame}{Sposób realizacji i przebieg prac}
\begin{enumerate}
  \item zbieranie danych z Twittera
  
  \item wybór i budowa narzędzi do ekstrakcji wiedzy
%  \begin{itemize}
%	\item klasyfikator sentymentu
%  \end{itemize}
  
  \item odkrywanie wiedzy
%  \begin{itemize}
%    \item obliczenie sentymentu wpisów
%	\item identyfikacja zwolenników i przeciwników
%	\item zastosowanie geolokacji
%	\item odkrywanie zachowań użytkowników 
%  \end{itemize}
  
\end{enumerate}

\center{
	\includegraphics[width=11cm]{img/gruby-model.png}
}
\end{frame}

% ==================================================================================
\section{Uzyskane rezultaty}
% ==================================================================================
\begin{frame}{Uzyskane rezultaty}
\begin{itemize}
  \item 7 mln tweetów  
  \item 49\% pozytywnych, 47\% negatywnych
\end{itemize}
\center{
	\includegraphics[width=9cm]{img/rozklad-wpisow.png}
}
\end{frame}

% ==================================================================================
\begin{frame}{Aktywność użytkowników i rozkład sentymentu}
\includegraphics[width=11cm]{img/tweety-w-meczu-nums.png}
\end{frame} 

% ==================================================================================
\begin{frame}{Interakcja między użytkownikami}
\begin{adjustwidth}{-5cm}{-5cm}
\begin{table}
\begin{tabular}{rl}
\includegraphics[width=0.50\textwidth]{img/reply-sentiment-zwolennik-zwolennik.PNG}
&
\includegraphics[width=0.50\textwidth]{img/reply-sentiment-przeciwnik-przeciwnik.PNG}
\end{tabular}
\end{table}
\end{adjustwidth}
\end{frame}

% ==================================================================================
%\begin{frame}{Struktura grup między meczami}
%\includegraphics[width=1\textwidth]{img/grupy-arsenal-nums.png}
%\end{frame}

% ==================================================================================
\begin{frame}{Aktywność kibiców, a odległość od stadionu}
\begin{adjustwidth}{-5cm}{-5cm}
\begin{table}
\begin{tabular}{rl}
\includegraphics[width=5.4cm]{img/odleglosc-od-stadionu-home.png}
&
\includegraphics[width=5.3cm]{img/odleglosc-od-stadionu-away.png}
\end{tabular}
\end{table}
\end{adjustwidth}
\end{frame}

% ==================================================================================
\begin{frame}{Inne eksperymenty}
\begin{enumerate}
  \item struktura grup między meczami
  \begin{itemize}
    \item zauważalna stała grupa kibiców
    \item im bardziej popularne mecze, tym mniejsze podobieństwo użytkowników
    w nich uczestniczących 
  \end{itemize}
  \item zmiana sentymentu w meczach
  \begin{itemize}
    \item gdy drużyna wygrywa -- wydźwięk bardziej pozytywny, gdy przegrywa --
    bardziej negatywny
  \end{itemize}
  \item struktura komunikacji
  \begin{itemize}
    \item przecwinicy najczęściej odpowiadają sobie nazwajem
    \item zwolennicy najczęściej podają swoje wpisy dalej (retweetują)
  \end{itemize}
  \item im kibice fizycznie bliżej siebie, tym częściej się ze sobą komunikują
\end{enumerate}
\end{frame}
% ==================================================================================
\begin{frame}{Podsumowanie eksperymentów}
\begin{itemize}
  \item użytkownicy sieci społecznościowych zachowują się w taki sam sposób
  jak w życiu codziennym
  \item najnowsze technologie komputerowe pozwalają na badanie dużych
  społeczności w sposób transparentny
  \item badanie sieci społecznych można zautomatyzować i zastosować również do
  innych dziedzin 
\end{itemize}
\end{frame}


% ==================================================================================
\appendix
% ==================================================================================
\begin{frame}{Podsumowanie pracy}
Osiągnięte cele:
\begin{itemize}
\item połączono geolokację, sentyment i sieci społeczne
\item zbudowano narzędzie do przetwarzania dużej ilości danych z~Twittera pod
kątem analizy sieci społecznych
\item odkryto ciekawe zależności wśród badanej społeczności
\end{itemize}
\vspace{0.5cm}
Możliwe kierunki rozwoju:
\begin{itemize}
\item rozbudowanie narzędzia do analizy sentymentu
\item odkrywanie tematów rozmów
\item bliższe przyjrzenie się grupom
\end{itemize} 
 
\end{frame} 
 
\end{document}
