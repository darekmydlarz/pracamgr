\chapter{Przegląd badań}
\begin{comment}
- dużo literatury, przegląd wiedzy dostępnej na dany temat
- sieci społeczne: SNA + analiza zachowań + grupy
- analiza sentymentu: techniki, itd
- twitter
- badania nad geolokacją, zastosowanie, jak
\end{comment}

W niniejszym rozdziale znajduje się aktualny stan badań dotyczący 4 tematów,
które składają się na tę pracę. Na początku opisana jest dziedzina sieci
społecznych, czym ta nauka się zajmuje, w jakich przypadkach może zostać
zastosowana.
Następnie omówiona zostaje analiza sentymentu wypowiedzi i przetwarzanie tekstu
celem ekstrakcji jego wydźwięku.
Później skupiam się nad temat związanym z geolokacją i opisem, co można dzięki
niej się dowiedzieć, a rozdział kończę omówieniem serwisu społecznościowego Twitter,
który został wykorzystany jako źródło danych do analizy sieci społecznych.








%%%%%%%%%%%%%%%%%%%%%%%%%%%%%%%%%%%%%%%%%%%%%%%%%%%%%%%%%%%%%%%% SIECI SPOŁECZNE
\section{Sieci społeczne}
Termin ten został użyty po raz pierwszy w 1954 roku przez Johna Arundela Barnesa
\cite{JABarnes}. Oznacza strukturę społeczną, którą tworzą jednostki (np. osoby
lub organizacje) i połączenia między nimi.
Analiza sieci społecznych jest badaną od wielu lat dziedziną nauki. Szybki
rozwój Internetu w XXI wieku wzbogacił ją o bogate źródło danych. Główne obszary
badań \cite{SNDAtopics} to między innymi:
\begin{itemize}
  \item statystyczna analiza sieci społecznych -- opisuje jak wygląda typowa sieć społeczna,
  badane są połączenia między jednostkami, aby sprawdzić czy posiadają kilka połączeń,
  czy sieć zbudowana jest z \emph{hubami}, czy może liczba połączeń rozłożona 
  jest równomiernie,
  \item odkrywanie grup/społeczności -- jest jednym z głównych tematów analizy 
  sieci społecznych; szukanie grup związane jest z klasteringiem i odkrywaniem 
  regionów sieci, które są odpowiednio gęste; problem powiązany jest z badaniem
  grafów, określaniem jak dzielić sieć na regiony,
  \item klasyfikacja wierzchołków - w niektórych sieciach część wierzchołków 
  może być oznaczona; badania skupiają się na tym by korzystając z atrybutów 
  danego wierzchołka móc ocenić jakie inne również mogłyby przyjąć daną etykietę,
  \item odnajdywanie ekspertów -- sieci społeczne mogą być używane jako narzędzia
  w celu odkrywania ekspertów do danego zadania,
  \item przewidywanie przyszłych połączeń wewnątrz sieci -- wiele badań skupia się
  na połączeniach wierzchołków celem odkrycia interesujących informacji na temat
  sieci społecznej; w wielu sieciach połączenia między węzłami są dynamiczne
  i wynikiem tych badań może być poprawna predykcja przyszłych połączeń
  węwnątrz sieci,
  \item ekstrakcja wiedzy z sieci -- polega na eksploracji danych z mediów 
  społecznościowych i eksploracji tekstu z serwisów społecznościowych; 
  eksploracja danych dostarcza naukowcom narzędzia do analizy dużych, 
  złożonych i często zmieniających się danych wewnątrz sieci, a eksploracja tekstu
  może prowadzić do odkrycia nowych połączeń między węzłami i nowych
  charakterystyk je łączących; jej użycie wpływawpłynąć na poprawę jakości
  badanej sieci. 
\end{itemize}



\subsection{Przykłady zastosowania sieci społecznych}
Wyniki badań nad sieciami społecznymi stwarzają wiele możliwości dla różnych
dziedzin życia. Ich zastosowanie może być zaaplikowane przez:
\begin{itemize}
  \item służby porządkowe -- policja może przy ich pomocy odkrywać powiązania
  między przestępcami i dochodzić do zależności między grupami przestępczymi,
  a także odkrywać, kogo dane grupy mogłyby zwerbować, 
  \item badania naukowe -- odkrywanie naukowców zajmujących się podobnymi tematami
  celem opracowania bardziej kompletnych wyników lub podjęcia nowego,
  wspólnego tematu,
  \item przedsiębiorstwa handlowe -- odkrywanie zbliżonych typów klientów i 
  oferwanie im produktów lub usług do nabycia przy użyciu systemów rekomendujących,
  \item służby zdrowotne -- użycie sieci społecznych może pomóc w określaniu
  obszarów, w które rozprzestrzeniają się wirusy gróźnych chorób, dzięki czemu
  możliwe może być zapobieganie ich dalszej ekspansji.
\end{itemize}



\subsection{Reprezentacja sieci społecznych}
Najczęściej spotykaną reprezentacją sieci społecznych jest reprezentacja grafowa.
W naturalny sposób modeluje ona jednostki jako węzły grafu i relacje między nimi
jako krawędzie. W zależności od rodzaju sieci graf taki może być skierowany lub 
nieskierowany oraz o krawędziach ważonych lub nieważonych. W przypadku
krawędzi ważonych waga danego połączenia może reprezentować na przykład
liczbę wiadomości wymienionych między węzłami. Kierunek krawędzi reprezentuje,
w która stronę dana komunikacja przebiegała. Wygląd takich grafów przedstawia rysunek 
\ref{image:graf-reprezentacja}.

\begin{figure}[ht!]
\centering
\includegraphics[width=90mm]{img/graf-reprezentacja.png}
\caption{Przykład grafu nieskierowanego bez krawędzi ważonych (po lewej)
oraz grafu skierowanego z krawędziami ważonymi (po prawej)}
\label{image:graf-reprezentacja}
\end{figure}




\subsection{Miary i pojęcia grafowe}
Zamodelowanie sieci społecznych w postaci grafów pozwala na skorzystanie z szeregu
miar związanych z tą dziedziną wiedzy. Dzięki nim możliwe jest odnajdywanie cech
charakterystycznych danej sieci. Najważniejsze miary pomagające odnaleźć najważniejsze
węzły to to \cite{estrada}: 


\subsubsection{Stopień wierzchołka}
Miara określająca liczbę krawędzi wchodzących i 
  wychodzących z wierzchołka (patrz rys. \ref{image:stopien-wierzcholka})
  
\begin{figure}[ht!]
\centering
\includegraphics[width=60mm]{img/stopien-wierzcholka.png}
\caption{Graf z oznaczonymi stopniami wierzchołków}
\label{image:stopien-wierzcholka}
\end{figure}

W przypadku grafów skierowanych możemy jeszcze mówić o stopniu wchodzącym 
(ang. \textit{in degree}) oraz wychodzącym (ang. \textit{out degree}) 
(patrz rys. \ref{image:stopien-wierzcholka-skierowany}).
  
\begin{figure}[ht!]
\centering
\includegraphics[width=75mm]{img/stopien-wierzcholka-skierowany.png}
\caption{Graf skierowany z oznaczonymi stopniami wierzchołków}
\label{image:stopien-wierzcholka-skierowany}
\end{figure}
  
  
\clearpage
\subsubsection{Pośrednictwo (ang. \textit{betweenness}) }  
Liczba najkrótszych ścieżek w grafie, które przechodzą przez dany węzeł podzielona
przez liczbę wszystkich najkrótszych ścieżek grafu. Przez najkrótszą ścieżkę 
rozumie się taką ścieżkę między dwoma węzłami grafu, dla której liczba krawędzi
jest najmniejsza.
  
\begin{equation}
BC(k) = \sum\limits_{i}\sum\limits_{j}\frac{\rho(i, k, j)}{\rho(i, j)}, \quad i \neq j \neq k
\end{equation}  

Przykładowo aby obliczyć wartość tej miary dla wierzchołka $B$ posłużmy się rysunkiem 
\ref{image:betweenness}.
Najkrótsze ścieżki między węzłami innymi niż $B$ to: $ABC,$ $ABD$, $ABE$, $CBD$, $CBE$, $DE$.
\mbox{W 5 z 6 z nich} znajduje się węzeł $B$, stąd wynika jego wartość \textit{betweenness}
równa $5/6$. 

\begin{figure}[ht!]
\centering
\includegraphics[width=75mm]{img/betweenness.png}
\caption{Najkrótsze scieżki przechodzące przez węzeł $B$}
\label{image:betweenness}
\end{figure}

Węzły o wysokiej wartości współczynnika \textit{betweenness} są interesujące
ponieważ mogą kontrolować przepływać informacji wewnątrz sieci oraz
mogą być zmuszone do przetwarzania większej ilości informacji.
Z tego wynika też, że mogą być skutecznym celem ataków.
    
  
\clearpage  
\subsubsection{Bliskość (ang. \textit{closeness})}  
Znormalizowana odwrotność sumy odległości między węzłami w grafie.
  
\begin{equation}
CC(i) = \frac{N - 1}{\sum\limits_{j}d(i, j)}
\end{equation}  

Dla każdej pary węzłów liczymy odległość między nimi (liczoną jako liczbę krawędzi),
a następnie dla każdego wierzchołka dzielimy tę wartość przez $N - 1$, gdzie $N$
to liczba wierzchołków. Przykładowy graf \ref{image:closeness} i tabela 
\ref{tab:closeness} z obliczeniami znajdują się poniżej.


\begin{figure}[ht!]
\centering
\includegraphics[width=75mm]{img/closeness.png}
\caption{Suma odległości do pozostałych węzłów w grafie}
\label{image:closeness}
\end{figure}
  

\begin{table}[ht!]  
\begin{center}  
\begin{tabular}{| c | c c c c c || c | c |}
 \hline
 & \multicolumn{5}{c ||}{Wierzchołki} & Suma odległości & Bliskość (\textit{closeness}) \\
 \hline
 & A & B & C & D & E &  $\sum\limits_{j}d(i, j)$ & $CC(i)$ \\
\hline
A & 0 & 1 & 2 & 2 & 3 & 8 & 0.5 \\ 
B & 1 & 0 & 1 & 1 & 1 & 4 & \textbf{1.0} \\ 
C & 2 & 1 & 0 & 2 & 2 & 7 & 0.57 \\ 
D & 2 & 1 & 2 & 0 & 1 & 6 & 0.67 \\ 
E & 3 & 1 & 2 & 1 & 0 & 7 & 0.57 \\ 
 \hline
\end{tabular} 
\end{center} 
\caption{Odległości między węzłami i wartości miary \textit{closeness}}
\label{tab:closeness}
\end{table}
  
Węzłem o najmniejszej sumie odległości do innych wierzchołków -- a co za tym idzie --
o największej wartości bliskości jest węzeł $B$. Wynika z tego, że jest to
wierzchołek najszybciej rozsyłający informacje wewnątrz sieci pomiędzy jej elementami. 


\clearpage
\subsubsection{Wektor własny (ang. \textit{eigenvector})}
Miara centralności węzła,  która oceniając dany węzeł bierze także pod uwagę 
wartości jego sąsiadów (bezpośrednio przyległych węzłów).
Zastosowanie tej wielkości pozwala wskazać najważniejszy węzeł w sytuacji,
gdy poprzednie miary zwracają równe wyniki. Wartość tej wielkości wyraża się
wzorem:

\begin{equation}
x_i = \frac{1}{\lambda}\sum\limits_{j=1}^nA_{ij}x_j
\end{equation}
gdzie $\lambda$ jest stała i równa największej wartości własnej macierzy sąsiedztwa
danego grafu, a wartość $A_{ij} = 1$, gdy wierzchołki mają wspólną krawędź,
w przeciwnym wypadku wynosi $0$. Przykładowe wartości wielkości \textit{eigenvector}
zaprezentowano na rysunku \ref{image:eigenvector}. Wartości te zostały
obliczone przy pomocy narzędzia Gephi\footnote{https://gephi.github.io/}.

\begin{figure}[ht!]
\centering
\includegraphics[width=40mm]{img/eigenvector.png}
\caption{Wartości wielkości \textit{eigenvector} w przykładowym grafie}
\label{image:eigenvector}
\end{figure}

Dla uzupełnienia terminologii związanej z traktowaniem sieci społecznej
jako grafu chciałbym przypomnieć jeszcze dwa pojęcia:
\begin{itemize}
  \item klika -- podgraf grafu, w którym wszystkie wierzchołki połączone są krawędzią
  \item k-klika -- klika składająca się z dokładnie $k$-wierzchołków. Na przykład
  k-klika o $k=3$ to podgraf zbudowany z 3 wierzchołków, gdzie między każdym z 
  nich znajduje się krawędź (patrz rys. \ref{image:klika}).
\end{itemize}

\begin{figure}[ht!]
\centering
\includegraphics[width=40mm]{img/klika.png}
\caption{Wierzchołki $ABC$ tworzą klikę o rozmiarze 3}
\label{image:klika}
\end{figure}








%%%%%%%%%%%%%%%%%%%%%%%%%%%%%%%%%%%%%%%%%%%%%%%%%%%%%%%%%%%%%%%%%%%%%% SENTYMENT
\clearpage\section{Sentyment wypowiedzi}
Recenzje, komentarze i opinie odgrywają istotną rolę w ocenie satysfakcji
z produktu lub usługi czy w badaniu reakcji na wydarzenia. Dane, które zawierają
takie informacje mają bardzo wysoki potencjał w odkrywaniu wiedzy.
Dowiadywanie się, co myślą inni ludzie zawsze było bardzo istotne w procesie
podejmowania decyzji. Internet dał nam możliwości zapoznania się z opiniami
innych zwykłych ludzi, ale także pozwolił na poznanie komentarzy 
ekspertów w swoich dziedzinach. Badanie sentymentu -- a więc wydźwięku
wypowiedzi (ocena wypowiedzi jako pozytywnej, negatywnej lub neutralnej)
-- odgrywa bardzo istotną rolę. Jak wynika z badań przeprowadzonych na ponad
2000 dorosłych Amerykanów \cite{pangLee} 81\% użytkowników Internetu
przynajmniej raz poszukiwało w Internecie informacji o jakimś produkcie
z czego od 73\% do 87\% osób twierdzi, że recenzje innych miały wpływ
na ich wybory.

Zastosowanie analizy sentymentu jest bardzo szerokie. Niektóre
z obszarów jej użycia to \cite{pangLeeApplication}:
\begin{itemize}
  \item portale internetowe z opiniami -- zastosowanie analizy sentymentu
może być użyte do poprawy błędów popełnionych przez użytkowników (gdy opinia
jest pozytywna, a użytkownik omyłkowo wybrał niską ocenę) lub gdy opinie
są ewidentnie stronnicze, mogą pomóc w faktycznej ocenie danego przedmiotu czy 
usługi
\item jako technologia wspomagająca większe systemy -- analiza sentymentu może
być wsparciem dla systemów rekomendacji; na przykład może służyć do 
nie rekomendowania produktów, które otrzymują negatywne opinie; 
w systemach serwujących reklamy kontekstowe, wykrycie pozytywnego sentymentu
na stronie może być powodem wyświetlenia jakiejś reklamy,
a wykrycie negatywnego sentymentu powodem jej ukrycia;
innym zastosowaniem jest ekstrakcja informacji, która może być
polepszona poprzez pomijanie zdań subiektywnych, zawierających sentyment
\item biznes -- poprzez dostarczenie informacji o odbiorze sprzedawanych produktów
i serwowanych usług; gdy na przykład sprzedawany laptop ma negatywny odbiór
stosując analizę sentymentu można to bardzo szybko wykryć i dowiedzieć się
dlaczego zaistniała dana sytuacja; firma może badać swój ogólny odbiór
w społeczeństwie -- szybko reagować na niezadowolenie klientów, lub wprowadzać
poprawki do swoich produktów; wykrywanie sentymentu może również pomóc
przewidzieć wyniki sprzedaży
\item polityka -- użycie analizy sentymentu jest wręcz naturalne dla tego obszaru
życia; partie czy politycy mogą badać odbiór społeczeństwa swoich programów
i decyzji; badanie sentymentu może im na przykład wskazać w jakich miejscach,
czy przy jakich postaciach się pokazać by zyskać sympatię wyborców; istotne
również mogą być informacje na temat rekacji społeczeństwa na planowane
przez rząd zmiany w prawie. 
\end{itemize}

Krótko mówiąc największym zyskiem związanym z badaniem sentymentu jest możliwość
zbadania opinii bardzo dużej liczby osób w sposób mechaniczny. Nie ma potrzeby
przeprowadzania ankiet, pytania ludzi co sądzą na dany temat. Internauci samodzielnie
przedstawiają swoje opinie w Internecie, a przy pomocy analizy sentymentu bardzo
łatwe staje się zbadanie nastrojów.


\clearpage

Badanie sentymentu nie jest trywialne. Związane jest bezpośrednio z 
przetwarzaniem języka naturalnego, które niesie ze sobą szereg problemów:

\begin{itemize}
  \item złożoność języka naturalnego -- bardzo trudnym zadaniem jest nauczenie 
programu komputerowego pełnego rozumienia języka naturalnego; co więcej każdy
język jest inny, więc dla każdego konieczne jest zastosowanie różnych
podejść -- inaczej trzeba zabrać się za badanie sentymentu w języku polskim
a inaczej w angielskim; język ciągle się rozwija, nie jest martwy,
\item trudność w analizie kontekstu wypowiedzi -- wykrycie ironii nie jest zadaniem 
prostym; bardzo często wypowiedzi mogą mieć związek z jakimś pojęciem 
zupełnie niezrozumiałym dla programu komputerowego, a oczywistym dla człowieka
(np. idiomy, odniesienia do wydarzeń na świecie)
\item slang w Internecie, skrótowce, literówki -- wszystkie te elementy
dodatkowo utrudniają analizę sentymentu; użytkownicy Internetu nie zawsze
dbają o jakość swojego języka, często stosują skróty, czy wyrażenia slangowe, 
które mogą być niezrozumiałe dla automatycznego analizatora sentymentu;
\item SPAM, szum -- wszystkie wpisy, które nie niosą ze sobą żadnej wartości
a pojawiają się w internetowych forach, serwisach z opiniami również stanowią
wyzwanie przy budowie narzędzia do analizy sentymentu.
\end{itemize}


\subsection{Techniki badania sentymentu}
Podejść do badania sentymentu jest wiele. Poniżej przedstawione są te, które
najlepiej nadają się do badania sentymentu na Twitterze (w związku z tym, że to
ten serwis jest źródłem danych w tej pracy), a które zostały opisane w artykule 
\cite{sentimentTechniques}. Oprócz nich przedstawię także metodę opracowaną
przez Alexandra Paka i Patricka Paroubek'a \cite{pakParoubekSentiment}, którą 
zastosowałem w swoich badanich. Techniki te to:

\subsubsection{Podejście oparte na słowniku (ang. \textit{lexicon based approach})}
Podejście polega na zastosowaniu słownika z wyrazami oznaczonymi jako pozytywne
i negatywne. Klasyfikator ocenia tekst na podstawie liczby wystąpień
odpowiednich słów. Niestety podejście to ma bardzo wysoki stopień błędów.
Przykładowa funkcja oceniająca sentyment słowa to:
\begin{equation}
X_t = \frac{p(pos | topic, t)}{p(neg | topic, t)}
\end{equation}
w tym przypadku wyrazy mają przypisany odpowiedni sentyment w zależności od tematu,
którego dotyczą.

Największym problemem tego podejścia jest brak mechanizmu radzenia sobie z kontekstem
słów.

\subsubsection{Naiwny klasyfikator Bayesa (ang. \textit{naive Bayes classifier})}
Jest to podejście probabilistyczne. W ramach tej metody zakłada się, że dana kategoria
tekstów $k_1$ (np. pozytywne) charakteryzuje się określonym słownictwem, 
a inna $k_2$ (negatywne) innym słownictwem. 
Na tej podstawie określamy prawdopodobieństwo jeszcze przed przeprowadzeniem
jakiejkolwiek klasyfikacji tekstu. Zakłada się także, że tekst, który posiada
słownictwo z kategorii $k_1$ w większej liczbie niż z kategorii $k_2$, powinien
być zaklasyfikowany do tej pierwszej. 
W tym przypadku jest to określenie klasyfikacji posiadając pewną wiedzę na temat
badanego tekstu.

Naiwny klasyfikator Bayesa opiera się na założenia o wzajemnej niezależności
słów. Oznacza to, że wyrazy, które identyfikują określoną kategorię mogą występować
niezależnie w różnych lub tym samym tekście. Taki naiwny klasyfikator może więc 
identyfikować i klasyfikować słowa, nie biorąc pod uwagę kontekstu w jakim one
występują. Pomimo, że jest to podejście naiwne, okazuje się skuteczne ze względu
na swoją prostotę. Wzór Bayesa określa bowiem prawdopodobieństwo tego, że szanse
przypisania tekstu do odpowiedniej klasy zależą od tego jak często jego słowa
należą do różnych klas i jak często do nich nie należą.

Krótko mówiąc, jeśli naiwny klasyfikator Bayesa w wybranym tekście znajdzie więcej
słów należących do klasy pozytywnej i jednocześnie mniej należących do negatywnej,
wówczas większe będzie prawdopodobieństwo zaklasyfikowania tekstu do pierwszej
kategorii. Klasyfikator ten zbiera uczy się klas wyrazów sukcesywnie analizując
kolejne teksty \cite{tomanekSentyment}.



\subsubsection{Technika maksymalnej entropii (ang. \textit{maximum entropy technique})}
Technika estymacji rozkładu prawdopodobieństwa. Główna zasada polega na tym,
że jeśli dane nie są dobrze znane, rozkład powinien być jak najbardziej jednolity,
to znaczy mieć maksymalną entropię. Do tej techniki mogą dochodzić ograniczenia,
które pozwalają by rozkład nie był maksymalnie jednolity. Ograniczenia
takie mogą pochodzić z oznaczonych już danych treningów i reprezentowane jako
oczekiwane wartości wybranych cech (wyrazów). 

Na przykład w jakimś przypadku
możemy założyć, że 50\% wpisów jest pozytywnych, wówczas pozostałe klasy
powinny posiadać po 25\% prawdopodobieństwa (negatywne, neutralne).
Taki model jest łatwy do zbudowania, ale staje się on bardziej skomplikowany
wraz z rosnącą liczbą ograniczeń. Jako cechy dodawane mogą być również
składniki wielowyrazowe zwiększające skuteczność tej techniki. Dlatego też
podejście to nie cierpi z powodu założenia o niezależności wyrazów.
Przykładowo wyrażenie ,,do widzenia'' może być traktowane jako całościowy term,
a nie jako każdy wyraz z osobna.

Niestety w związku z tym, że ograniczenia pochodzą z danych treningowych,
jest duża szansa, że dane te będą relatywnie rzadkie i metoda ta może prowadzić
do przeuczenia.


\subsubsection{Maszyny wektorów nośnych (ang. \textit{support vector machines})}
Support vector machines to podejście stosujące duży margines między klasami.
Główna idea polega na znalezieniu hiperpłaszczyzny, która podzieli teksty na pozytywne
i negatywne z marginesem pomiędzy klasami tak dużym jak to tylko możliwe.
Technika ta zbudowana jest na zasadzie strukturalnej minimalizacji ryzyka 
(ang. \textit{structural risk minimization principle}). Celem jest znalezienie
funkcji $h$, dla której błąd klasyfikacji losowego tekstu będzie jak najmniejszy.
Oznaczając hiperpłaszczyznę przez $\vec{h}$, a tekst przez $\vec{t}$ oraz klasy, do
których może trafić jako $C_j \in \{1, -1\}$ wówczas możemy zapisać to 
postaci:
\begin{equation}
\vec{h} = \sum\limits_{i}\alpha_iC_i\vec{t_j}, \quad \alpha_i \geq 0
\end{equation}

Wartość $\alpha_i$ może być znaleziona przez rozwiązanie problemu podwójnej 
optymalizacji. Teksty o $\alpha_i$ większym od zera, to te które biorą udział
w szukaniu funkcji $h$ nazywa się je wektorami wspierającymi 
(ang. \textit{support vectors}).

Wybór cech (wyrazów) jest bardzo ważnym zadaniem w technikach uczenia maszynowego.
Musi to zostać tak wykonane by uniknąć przeuczenia i jednocześnie zwiększyć
ogólną dokładność. Maszyny wektorów nośnych mają wysoki potencjał radzenia
sobie z dużą liczbą wymiarów. Mierzą złożoność hipotezy którą dzielą dokumenty, 
a nie liczbę cech. W związku z tym liczba cech nie jest problemem.
Technika ta radzi sobie z duża liczbą słów poprzez oznaczanie części z nich jako
nieistotne (tych najrzadziej pojawiających się). Niestety czasami prowadzi to 
do utraty informacji. 

Chociaż SVM przewyższa wszystkie tradycyjne metody klasyfikacji sentymentu,
to niestety jest czarną skrzynką. Trudne jest zbadanie natury klasyfikacji i
zidentyfikowanie, które słowa są dla niej istotne. Jest to jedna z głównych wad
korzystania z tej techniki do klasyfikacji tekstów. 

\subsubsection{Metoda Alexandra Paka i Patricka Paroubek'a}
\label{subsubsection:pakandparoubek}
Technika jest odpowiedzią na problemy związane z brakiem odpowiedniego słownika
do oceny sentymentu. Została opracowana z uwzględnieniem Twittera i korzysta
w związku z tym z pewnych założeń. Skoro nie ma żadnego idealnego słownika
ze słowami oznaczonymi jako pozytywne lub negatywne, to trzeba go mechanicznie
zbudować. Do budowy takiego leksykonu zostały wykorzystane wpisy na Twitterze,
które zawierają emotikony podzielone na pozytywne (np. \texttt{:)}) i 
negatywne (np. \texttt{;(}). 

Następnie spośród ściągniętych wpisów z Twittera analizowane są te,
które zawierają odpowiednie emotikony i zliczana jest liczba wystąpień
każdego wyrazu w każdym ze zbiorów (pozytywnym i negatywnym).
W wyniku tego budowany jest leksykon zawierający wyrazy wraz z liczbą
ich wystąpień w każdej z klas.
W związku z tym, że wpisy na Twitterze ograniczone są do 140 znaków, autorzy przyjęli
założenie, że emotikona dotyczy całego wpisu. 
Ocena tekstu $T$ składającego się z wyrazów $w_1, w_2, \ldots w_n$ obliczana 
jest jako:
\begin{equation}
valence(T) = \frac{\sum\limits_{i = 1}^n valence(w_i)}{n}
\end{equation}

gdzie wartość $valence(w_i)$ obliczana jest przy zastosowaniu skonstruowanego
leksykonu i równa delta IDF (ang. \textit{inverse document frequency} -- 
powszechnie stosowana miara ważności słowa w oparciu o liczbę wystąpień):
\begin{equation}
valence(w_i) = log\frac{N(w_i, M^+) + 1}{N(w_i, M^-) + 1}
\end{equation} 
Zastosowanie takiego wzoru prowadzi do tego, że niezależnie jak często
dany wyraz pojawił się w zbiorze treningowym, najważniejsza jest jego polaryzacja.
Gdy na przykład słowo \textit{świetny} pojawiło się w zbiorach pozytywnym
i negatywnym odpowiednio 1000 i 20 razy, a słowo \textit{przezacny} odpowiednio
50 i 1 raz to ich wpływ na ocenę tekstu będzie identyczny.






%%%%%%%%%%%%%%%%%%%%%%%%%%%%%%%%%%%%%%%%%%%%%%%%%%%%%%%%%%%%%%%%%%%%%% GEOLOKACJA
\clearpage\section{Geolokacja}
Geolokacja to identyfikacja położenia geograficznego jakiegoś obiektu.
Może odnosić się do procesu zdobywania takiej informacji lub do już zdobytej
wiedzy na ten temat. Główne sposoby pozyskiwania takich danych to:
\begin{itemize}
  \item korzystanie z urządzeń GPS -- wbudowanych we współczesne telefony
komórkowe, tablety, itp.,
  \item pozycjonowanie względne -- ustalanie pozycji na podstawie bazowych stacji
telefonii komorków, ruterów WI-FI,
\item użcyie bazy adresów przypisanych do IP.
\end{itemize} 
Zastosowanie geolokacji może być bardzo szerokie, między innymi:
\begin{itemize}
  \item dostarczanie lokalnych wiadomości
  \item dystrybucja treści cyfrowych -- może być np. blokowana możliwość kupna, 
  dla niektórych lokalizacji
  \item wyszukiwanie lokalnych usług, przedsiębiorstw
  \item wyświetlanie zlokalizowanych reklam
  \item zapobieganie nadużyciom zakupowym -- sprawdzenie geolokacji
  klienta sklepu internetowego i porównanie jej z danymi z karty kredytowej,
  w celu ochrony osób, którym na przykład taka karta została skradziona
  \item prezentowanie różnych treści na stronach w zależności od lokalnego
  prawego (np. ukrywanie treści zabronionych w danym miejscu).
\end{itemize}
W szczególności w przypadku sieci społecznych geolokacja może być pomocna do
ustalenia miejsca przebywania danych grup i może prowadzić
do uzupełnienia zebranych danych o kolejne, wzbogacające analizę danej społeczności,
pozwalające na wyciągnięcie bogatszych wniosków. Pomocnym może być na przykład
zbadanie reakcji społeczeństwa w różnych regionach kraju na planowane zmiany
w prawie przez rząd -- i może to prowadzić albo do ich wprowadzenia albo wycofania.

Jako główne zalety stosowania geolokacji z punktu widzenia użytkowników
telefonów komórkowych, sieci społecznych (\cite{lostInGeolocation}
to dzielenie się ze społecznością (56\%) oraz dzielenie się z osobami, które znają
lub mogą spotkać (41\%). Głównymi problemami, przed dzieleniem się geolokacją
są obawy o prywatność (33\%) oraz brak korzyści, zainteresowania (26\%).



%%%%%%%%%%%%%%%%%%%%%%%%%%%%%%%%%%%%%%%%%%%%%%%%%%%%%%%%%%%%%%%%%%%%%% TWITTER
\clearpage\section{Twitter}
Twitter to serwis społecznościowy o charakterystyce mikrobloga zorientowany
na szybką i bezpośrednią komunikację. Pozwala on
na umieszczanie wpisów nie dłuższych niż 140 znaków. Domyślnie wszystkie 
wpisy są publiczne, a użytkownicy mają możliwość publicznej wymiany zdań
z innymi. Każdy użytkownik ma możliwość wyboru użytkowników, których
wpisy chce widzieć na swojej stronie głównej.

Podstawowe pojęcia związane z tym serwisem to:
\begin{itemize}
  \item śledzenie (ang. \textit{follow}) -- osób, organizacji; śledzenie jakiegoś
  użytkownika oznacza wyświetlanie wszystkich jego wpisów na swojej stronie
  głównej 
  \item tweet -- pojedynczy wpis/post na Twitterze; maksymalnie długość to 140
  znaków; może dodatkowo zawierać zdjęcie lub informację o geolokalizacji,
  \item retweet -- oznacza przekazanie jakiegoś wpisu dalej; jeśli użytkownik A
  śledzi użytkownika B i użyje funkcji retweet dla jednego z jego wpisów, 
  wówczas osoby śledzące użytkownika A, również zobaczą ten wpis na
  swojej stronie głównej,
  \item odpowiedź (ang. \textit{reply}) -- odpisanie na jakąś wiadomość w serwisie
  Twitter, skomentowanie jej; serwis łączy takie wpisy w jedną grupę, wyświetlając
  je jeden obok drugiego,
  \item newsfeed -- inaczej strona główna użytkownika, na której widzi wszystkie
  tweety wysłane przez osoby, które śledzi,
  \item hashtag -- użycie symbolu \# wraz z jakimś słowem, ułatwia
  rozmowy na wspólne tematy, wśród większych grup użytkowników 
  (np. \#worldcupfinal dla osób komentujących finał mistrzostw świata).
\end{itemize} 

\subsection{Twitter jako źródło danych}
Twitter używany jest przez 140 milionów użytkowników wysyłających ponad 400 
milionów wpisów dziennie. Szybkość komunikacji i łatwość publikacji wpisów
sprawia, że staje się medium komunikacyjnym dla wielu grup ludzi.
Odgrywał ważną rolę w wydarzeniach społeczno-politycznych,
takich jak Arabska Wiosna w 2010, czy okupowanie Wall Street w 2012 
(\cite{TwitterDataAnalytics2013}).
Serwis ten jest również bardzo często wykorzystywany do komentowania wydarzeń
sportowych. W trakcie mundialu w Brazylii użytkownicy wysłali 672 miliony
wpisów z tagiem \#WorldCup \cite{TwitterStatsWorldCup}.

Popularność Twittera jako źródła informacji doprowadziła do rozwoju badań
w różnych dziedzinach. Pomoc humanitarna i w przypadku klęsk żywiołowych
jest jedną z domen, gdzie informacje z Twittera są używane w celu zapewnienia
odpowiedniej pomocy. Naukowcy wykorzystują go by przewidzieć występowanie trzęsień
ziemi i określić odpowiednich użytkowników, których śledzenie dostarcza informacji
związanych z katastrofą \cite{TwitterDataAnalytics2013}.

Dane z Twittera można uzyskać poprzez API udostępniające 1\% wpisów.
Pozyskanie większej ich liczby wiąże się z dodatkowymi opłatami, z których
korzystają największe przedsiębiorstwa badające społeczność Twittera.
Korzystając ze Streaming API mamy możliwość konsumowania na żywo
wpisów spełniających podane kryteria wyszukiwania (np. słowa kluczowe, lokalizację).
Serwis udostępnia również REST API, które służy do pozyskiwania statycznych danych
-- wpisów użytkownika w momencie wysłania żądania, listy jego obserwowanych,
czy szczegółowych danych go dotyczących. Tweety zawierające dane o lokalizacji
są do klienta przesyłane wraz z tą informacją, dzięki czemu możliwe jest
skorzystanie z nich w analizach.

