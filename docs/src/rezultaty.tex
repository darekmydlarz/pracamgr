\chapter{Analiza zachowań użytkowników serwisów społecznościowych}

\begin{comment}

\end{comment}
[przykładowe rezultaty działania]

[przykłady działania i ewentualne błędy i dyskusja ograniczeń na podstawie
przykładów]

- opis badanej grupy 
- kibice piłkarscy, anglojęzyczni, 
- 8 mln tweetów, chelsea,
- manchester united, manchester city, arsenal
- głównie Wielka Brytania, USA, Archipelag Malejski, Nigeria, Ghana, Kenia,
Malezja, Singapur, Indonezja, Indie, RPA

\section{Analiza sentymentu}
- bardziej pozytywny, gdy drużyna wygrywa, gdy padają bramki
- spodziewane, intuicyjne wyniki badania sentymentu
\section{Sentyment a geolokalizacja}
- bardziej pozytywny sentyment w mieście drużyny zwycięskiej
- kibice angielscy umiarkowanie zadowoleni
- raczej sentyment w okolicach neutralnego
- kibice amerykańscy, afrykańscy czy malezyjscy cechują się wyższą wartością
sentymentu
\section{Sieci społeczne}
[grupy użytkowników, którzy użytkownicy skupiają na sobie uwagę, jaki niosą sentyment,
jaki generują w ramach swojej grupy]

- najczęściej uwagę (najwyższa wartość miary `authority`, najwyższa wartość
inDegree) skupiają 
oficjalne profile drużyn, które zazwyczaj niosą sentyment neutralny 
-- lekko pozytywny 

- liczba użytkowników biorących udział w wielu wydarzeniach jest
odwrotnie proporcjonalna to liczby tych wydarzeń (im więcej wydarzeń, tym
mniej użytkowników wzięło udział w nich wszystkich)

- największą uwagę skupiają spotkania z największymi rywalami