\chapter{Wstęp}

\begin{comment}
- motywacja, po co to robimy
- state of the art. Jak wyglada dziedzina, ktora sie zajmujemy?
Stosujemy tu metode szybkiego top down z ogolnikow do szczegolow.
Przechodzimy do tematyki, ktora sie zajmujemy. Piszemy co zrobiono precyzyjnie
w danej tematyce. Cytujemy artykuly (ksiazki - won!) Piszemy to w celu ..
- .. zdefiniowania celow pracy
- a potem jak te cele chcemy osiagnac, jakimi metodami
- potem piszemy co bedzie znajdowac sie w kazdym z rodzialow
- i na koncu jaki jest impact pracy
\end{comment}

W dzisiejszych czasach wpływ Internetu na życie codzienne jest niepodważalny.
Już od kilkunastu lat świat globalnej wioski przenika się z życiem realnym.
Nikogo nie dziwią prezentowane w kanałach informacyjnych komentarze
pochodzące z sieci, których autorami są zarówno osoby znane jak i zwykli
internauci. Rozrost Internetu przebiega w błyskawicznym tempie, a wydarzenia na
świecie komentowane są na żywo przez wielu ludzi. Aktualne trendy tworzone są na
blogach, mikroblogach czy serwisach społecznościowych.

Wyzwanie wobec ogromu tych informacji podejmuje dzisiejsza informatyka.
Przetwarzanie tak dużej ilości danych wymaga wielu zautomatyzowanych procesów.
W dzisiejszych czasach nie wystarczy już dowiedzieć się kto z kim najczęściej
się komunikuje, ale dużo bardziej interesujące jest to, o czym dany internauta
pisze i w jaki sposób to czyni.

Wielkie firmy chcą wiedzieć jak odbierane są ich produkty, jakie emocje
wzbudzają wśród klientów ich usługi i czy udaje im się spełniać ich oczekiwania.
Analiza użytkowników serwisów społecznościowych może być także bardzo
interesującym przedmiotem badań socjologów nad zmieniającym się społeczeństwem i
wpływem Internetu na ten proces.
Dodatkowo, analiza geolokalizacji może pozwolić marketingowcom na
odkrywanie nowych rejonów świata, w których mogliby oferować swoje
produkty i usługi.

Naprzeciw tym potrzebom budowane są systemy informatyczne, które potrafią takie
informacje uzyskać, przetwarzać i prezentować. Przykład takiego systemu został
zrealizowany w ramach tej\linebreak pracy~magisterskiej.

\section{Cel pracy}
Niniejsza praca skupia się na analizie zachowań użytkowników w wybranych
portalach społecznościowych. Przedmiotem badań są użytkownicy serwisu
mikroblogowego Twitter. W ramach pracy staram się odpowiedzieć na pytania:
\begin{itemize}
  \item jak internauci korzystają z mediów społecznościowych,
  \item kiedy są najaktywniejsi,
  \item jakie wyrażają emocję,
  \item z jakich miejsc komentują,
  \item czy i w jakie grupy się łączą.
\end{itemize}

Trzeba pamiętać, że analiza serwisów społecznościowych niesie ze sobą wiele 
wyzwań, z którymi trzeba się zmierzyć. Są to między innymi przetwarzanie
języka naturalnego -- który zawiera wiele skrótów, wyrażeń slangowych, błędów
ortograficznych i typograficznych, sklejanie wyrazów, używanie słów zapożyczonych
z innych języków, itp, radzenie sobie z ogromną ilością przetwarzanych informacji,
analizowanie dużej liczby krótki wiadomości, czy danych zaszumionych.
W związku z powyższym zebrane dane muszą być odpowiednio przetworzone i przefiltrowane
zanim zostaną przeprowadzone na nich jakiekolwiek operacje.

W ramach tej pracy pobrałem z serwisu Twitter w ciągu 3 miesięcy blisko 8
milionów wpisów, opracowałem metodę analizy sentymentu~-- czyli wydźwięku
wypowiedzi (pozytywna, negatywna lub neutralna), dokonałem badania sieci
społecznych i analizy zebranych danych poprzez połączenie ze sobą jednocześnie
wykrywania sentymentu, badania użytkowników i użycia geolokacji.

\section{Zawartość pracy}
W niniejszy dokumencie można wyróżnić 3 części. W pierwszej prezentuję
aktualny przegląd badań dotyczący poruszanych przeze mnie tematów, który
opisany został w rozdziale \ref{chapter:przegladbadan}. Można tam znaleźć
informacje na temat sieci społecznych, analizy sentymentu, geolokacji
a także o Twitterze w kontekście aktualnego stanu wiedzy na ich temat i sposobu
wykorzystania ich w nauce.

Przez drugą część można rozumieć rozdziały \ref{chapter:koncepcjarozwiazania}
oraz \ref{chapter:architektura}, gdzie prezentuję odpowiednio koncepcję
rozwiązania oraz architekturę systemu. W pierwszym rozdziale
przedstawiam koncepcję, metody i algorytmy, które zastosowałem w procesie
przeprowadzania badań. Omawiam tam również sposób w jaki dane były zbierane
i przetwarzane. Opisany jest algorytm badania wydźwięku wypowiedzi a także
sposoby badania sieci społecznych i geolokacji.

Trzecia część pracy to rozdział \ref{chapter:eksperymenty}, w którym
prezentowane są przeprowadzone eksperymenty oraz ich wyniki.
Każdy eksperyment jest opisany, posiada w miarę potrzeb załączony wykres,
oraz wnioski, które można na jego podstawie wysnuć.
