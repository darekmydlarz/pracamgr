\chapter{Opis systemu}
\label{chap:opisSystemu}
\section{Portale społecznościowe}
Portale społecznościowe to serwisy internetowe, których działanie oparte jest na
interakcjach pomiędzy ich użytkownikami. Najczęściej użytkownicy tych usług mogą
tworzyć grupy znajomych i dzielić się informacjami między sobą. Portale
społecznościowe ułatwiają komunikację między ludźmi z wielu zakątków świata
gromadząc ich w jednej usłudze. Przy ich pomocy odnalezienie się osób, które
znają tylko swoje podstawowe dane, staje się prostym zadaniem.

Do jednych z najpopularniejszych tego typu serwisów należą Facebook
(niekwestionowany lider) -- z ponad 1 miliardem aktywnych
użytkowników\footnote{http://techcrunch.com/2013/07/24/facebook-growth-2/},
Twitter -- 250 mln aktywnych
użytkowników\footnote{http://techcrunch.com/2014/04/29/twitter-beats-in-q1-with-250m-in-revenue-and-picks-up-14m-new-monthly-active-users/},
LinkedIn -- 260 mln\footnote{Hempel, Jessi (July 1, 2013). "LinkedIn: How It's
Changing Business". Fortune. pp. 69–74.} czy Google+ z 300
mln\footnote{http://marketingland.com/google-hits-300-million-active-monthly-in-stream-users-540-million-across-google-63354}
. O ile niektóre serwisy nie są nastawione na żaden specyficzny rodzaj
aktywności swoich użytkowników (jak najwięksi gracze: Facebook, Twitter czy
Google+), o tyle część z nich ukierunkowuje się na wyspecjalizowanych odbiorców,
oferując przy tym specjalne usługi. Do ich grona można zaliczyć między innymi
serwisy LinkedIn (profile zawodowe użytkowników -- ułatwianie poszukiwania
pracy, rozwoju zawodowego), Endomondo (serwis dla sportowców, posiadający
aplikację mobilną do mierzenia swoich wyników), Flickr (skupiający społeczność
fotografów prezentujących swoje dzieła), Last.fm (dla muzyków i osób
interesujących się muzyką) i wiele, wiele innych wyspecjalizowanych serwisów.

Liczba użytkowników portali społecznościowych stale rośnie. Pokolenie, które
urodziło się w dobie powszechnego internetu korzysta z niego w swobodny sposób,
dzieląc się swoimi przemyśleniami, zdjęciami czy emocjami. W związku z tym, że
co raz więcej osób udziela się w tego typu serwisach stają się one ważnym
przedmiotem badań dla naukowców. Nigdy wcześniej odkrywanie zachowań
społeczności ludzkich nie było tak proste. Już nie ma potrzeby przeprowadzania
ankiet bezpośrednich pytając o zachowania czy relacje z innymi, a wystarczy
dostęp do danych z serwisów społecznościowych, aby móc zauważyć ciekawe rzeczy z
punktu widzenia socjologii.

Na ich podstawie można dowiedzieć się jak zachowują się ludzie, w jakie grupy
się łączą i dlaczego -- np.
wspólne pasje, zainteresowania, tematy bieżące (ślub, studia, wydarzenia na
świecie), jaki jest czas egzystencji takich grup, jak one ewoluują i dlaczego i
jaki ma to związek z życiem codziennym ich członków, czy wydarzeniami na
świecie. Niektóre grupy są gęstsze i bardziej trwałe (np. kibice danej drużyny
sportowej), a inne luźne i szybko wymierające (np. grupa komentatorów jakiegoś
pojedynczego wydarzenia na świecie). Przy pomocy mediów społecznościowych
możliwe jest zauważenie pewnych charakterystycznych zachowań odpowiednich grup,
co może prowadzić do prób zaspokojenia ich potrzeb, poprzez serwowanie np.
bardziej dopasowanych reklam, czy nawet działania bardziej konkretne -- jak np.
wprowadzenie jakiegoś produktu na dany rynek, w którym nie jest on dostępny, a
jest nim w tym miejscu duże zainteresowanie. Oprócz tych pozornie błahych
zastosowań, portale społecznościowe mogą mieć dużo większą moc. Na przykład
podczas zamieszek w Turcji w 2013 roku ludność tego kraju intensywnie
wykorzystywała Twittera do informowania o wydarzeniach, czy umawiania się na
zebrania, demonstracje, i tym podobne -- co de facto doprowadziło do zablokowania
tego serwisu w tym kraju. Usługi te mogą mieć więc także swój duży udział w
przemianach politycznych świata i dlatego warto się nimi interesować i je badać.

\subsection{W jaki sposób można badać portale społecznościowe}
Aby móc badać zachowania użytkowników portali społecznościowych na samym
początku niezbędne jest posiadać dane z tychże usług. Najpopularniejszą formą
ich zdobywania jest skorzystanie z publicznego API\footnote{API -- Application
Programming Interface -- interfejs programowania aplikacji, ściśle określony
zestaw reguł w jaki programy komunikują się miedzy sobą
(http://pl.wikipedia.org/wiki/Application\_Programming\_Interface)} serwisów
społecznościowych. W większości przypadków udostępniane dane są limitowane. Na
przykład Twitter i Facebook wymagają wygenerowania unikalnego klucza dla każdej
aplikacji klienckiej i na tej podstawie część danych jest dostępna do pobrania.
Innym sposobem dostępu do danych, jest napisanie programu, który jest crawlerem
(czyli programem służącym do analizowania struktury dokumentów HTML
stron/serwisów internetowych) zbierającym dane na temat wybranego serwisu. Dużym
minusem tego rozwiązania jest to, że jego zaprogramowanie jest większym
wyzwaniem dla programisty, a także fakt, iż nie jest on odporny na zmiany
wyglądu stron internetowych i musi być w związku z tym często poprawiany.
Dodatkowo, w wielu przypadkach ma on dostęp do mniejszej ilości danych, niż ma
to miejsce w przypadku korzystania z oficjalnych API. Istnieje jeszcze jeden
sposób pozyskiwania takich danych -- można je po prostu kupić. Ich sprzedażą
zajmują się wyspecjalizowane serwisy, często mające umowę z portalami
społecznościowymi (przykładem takiej usługi jest GNIP).

Posiadając już odpowiednie dane można przejść do ich analizy. Można ją
przeprowadzić korzystając z ogólnie dostępnych programów (np. Gephi, CFinder) a
także samodzielnie pisząc własne algorytmy i narzędzia służące do wyszukiwania i
przetwarzania interesujących informacji.

\subsection{Zastosowanie sentymentu i geolokacji w analizie
zachowań użytkowników portali społecznościowych}

Dotychczasowe badania nad zachowaniami użytkowników portali społecznościowcych
ukierunkowane są na analizę relacji między nimi - pokazują, kto jest autorytetem
w danej grupie, kto skupia na sobie największą liczbę osób, przez kogo
informacje mogą najłatwiej i najszybciej dotrzeć do reszty.

W niniejszej pracy dodatkowo została przeprowadzona analiza sentymentu. Dzięki
niej można w lepszy sposób ocenić czy i dlaczego tworzą się konkretne grupy
użytkowników - często osoby prezentujące ten sam wydźwięk wypowiedzi komunikują
się między sobą.

Wzbogacenie takiego modelu o informacje o geolokacji również staje się źródłem
kolejnych wniosków - ludzie z jednego miejsca na Ziemii mogą wypowiadać się na
dany temat w zupełnie inny sposób, niż pozostali. Połączenie ze sobą tych trzech
obszarów badań pozwala uchwycić lepszy obraz danej sytuacji i wyciągnąć lepsze
wnioski, bardziej poprawne oraz mniej podatne na błędy.

\newpage \section{Budowa systemu}
\begin{figure}[ht!]
\centering
\includegraphics[width=120mm]{img/budowa-systemu.png}
\caption{Gruboziarnisty model systemu}
\label{image:model-systemu}
\end{figure}
- gruboziarnisty schemat systemu

- opis modelu

\subsection{Opis modułów systemu}
[z uwzględnieniem novum]

- twitter listener (słowa kluczowe)

- sentiment classifier (paroubek, dokładnie, dlaczego takie podejście)

- geodata visualizer (google maps, open street maps, cartodb)

- network analyser (gephi, cfinder) - omówić wykorzystanie

- novum: połączenie tych trzech obszarów w jeden

\subsection{Narzędzia użyte do budowy systemu}
- PostgreSQL

- Maven

- Git

- Java (Crawler - Twitter4j, Classifier, Geodata - REST API)

- IntelliJ IDEA

- pgAdmin III

- Play Framework

- Google Charts / JavaScript

- Google Maps, Open Street Maps, CartoDB Maps 


\newpage \section{[Dyskusja wyboru rozwiązania]}
- Twitter -- publiczne wypowiedzi, nastawaiony na komentowanie wydarzeń na żywo,
wspiera geolokalizację

- Paroubek Sentiment Classifier -- krótkie wypowiedzi, świetnie się do nich
nadaje (porównać do innych rozwiązań, dlaczego nie gotowy słownik,
dlaczego nie Naive Bayes), jak zastosowano negację, co ze słowami kluczowymi,
zastosowanie emotikon

- Gephi - otwarty, posiada wiele algorytmów

- CFinder - ułatwia szukanie grup

- Google Charts / Maps - wygodne, dobre API do do wykresów i map

- Open Street Map - otwarty projekt, API do wyciągania informacji o miejscu
na podstawie współrzędnych geograficznych

- moje rozwiązanie jest interesujące ponieważ łączy 
trzy ciekawe obszary w jeden, dotąd nie łączone

\subsection{Ograniczenia, porównania z innymi podobnymi systemami, złożoność}
- nie każdy ma Twittera -- badanie obarczone jest pewnym błędem,
jeśli chciałoby się jego wyniki przypisać całej społeczności

- niewiele osób włącza geolokalizację (domyślnie wyłączona)

- badanie języka jest bardzo trudne (ironia, ukryte żarty, spam, slang, 
szybko zmienijący się zasób słów, okresowa moda na różne określenia, 
skróty, błędy typograficzne, ortograficzne, słowa z obcych języków)


- klasyfikator sentymentu mimo usunięcia słów kluczowych ściśle związany
ze zbiorem uczącym i charakterystyką Twittera - tj. krótkich wypowiedzi 
(do dłuższych recenzji książek, produktów nie spełniłby swojego zadania)