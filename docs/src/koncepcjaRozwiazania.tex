\chapter{Koncepcja rozwiązania}

W niniejszym rozdziale skupiam się na opisie koncepcji rozwiązania wykorzystania
analizy sentymentu, sieci społecznych i geolokacji w analizie zachowań
użytkowników serwisów społecznościowych. W kolejnych podrozdziałach opisuję
sposoby w jaki dane zagadnienia zostały zaaplikowane w moich badaniach.
Na początku przedstawiam gruboziarnisty model systemu (\ref{section:modelsystemu}),
który prezentuje jego najważniejsze moduły. Następnie opisuję tematykę
i sposób gromadzenia danych (\ref{section:gromadzeniedanych}) potrzebnych do 
przeprowadzenia analizy internautów. Później omawiam metodę jaką zastosowałem 
podczas analizy sentymentu (\ref{section:analizasentymentu}) wpisów użytkowników.
W dalszej kolejności skupiam się nad zastosowanymi
sposobami analizy sieci społecznych (\ref{section:siecispoleczne})
i całość kończę omówieniem wykorzystania geolokacji w moich badaniach 
(\ref{section:wykorzystaniegeolokacji}).








%%%%%%%%%%%%%%%%%%%%%%%%%%%%%%%%%%%%%%%%%%%%%%%%%%%%%%%%%%%%%%%% MODEL SYSTEMU
\section{Model systemu}
\label{section:modelsystemu}
Stworzony system zbudowany jest z kilku modułów. Wyróżnić można jego
trzy główne części:
\begin{itemize}
  \item gromadzenie danych,
  \item przetwarzanie danych,
  \item analiza zebranych danych.
\end{itemize} 
Wszystkie informacje zapisywanie są w jednej, centralnej bazie danych.
Schemat systemu zaprezentowany jest na rysunku 
\ref{image:gruboziarnisty-model-systemu}.


\clearpage

\begin{figure}[ht!]
\centering
\includegraphics[width=140mm]{img/budowa-systemu.png}
\caption{Gruboziarnisty model systemu}
\label{image:gruboziarnisty-model-systemu}
\end{figure}

Każda z wyróżnionych powyżej części systemu bierze udział w innym etapie
badań. Na początku najważniejsze jest gromadzenie danych, po którym następuje
ich przetwarzanie by na końcu zająć się analizą i próba ekstrakcji wiedzy.
Wszystkie te części są ze sobą połączone wspólną bazą danych, w której
przechowywana jest zgromadzona wiedza i wyniki przetwarzania i analizy danych.
Poniżej pokrótce omówione są wszystkie z tych części.  

%%%%%%%%%%%%%%%%%%%%%%%%%%%%%%%%%%%%%%%%%%%%%%%%%%%%%%%%%%%% GROMADZENIE DANYCH
\section{Gromadzenie danych}
\label{section:gromadzeniedanych}
W tej sekcji opisuję sposób przeprowadzenia początkowych prac związanych
z moimi badaniami, które związane były ze zgromadzeniem danych potrzebnych
do przeprowadzania analiz. Etap ten był niezwykle istotny i przeprowadzenie
go, umożliwiło dalsze prace. Zgromadzone dane pochodzą z serwisu Twitter
i zostały uzyskane przy pomocy udostępnionego publicznie API (interfejsu
programistycznego).

\subsection{Tematyka danych}
% piłka nożna, kibice
Aby przeprowadzić analizę danych koniecznym było wybranie podzbioru
użytkowników Twittera. Dwa aspekty zadecydowały o tym, że skupiono się nad
analizą anglojęzycznego środowiska piłkarskiego:
\begin{itemize}   
  \item struktura językowa użytkowników sieci Twitter -- według badań firmy Gnip
  \cite{GnipTwitterLanguages} (zajmującej się gromadzeniem danych z tego serwisu
  społecznościowego) w 2013 roku ponad 50\% tweetów wysłanych zostało w języku
  angielskim.
  Dla porównania w języku polskim było to zaledwie 0.11\%.
  Dlatego też badanie użytkowników anglojęzycznych ma największy sens, gdyż
  prowadzi do zebrania największej liczby tweetów.
  Wybór konkretnego języka komunikacji jest o tyle istotny, iż wpływa znacznie
  na część badań związaną z analizą sentymentu,
  \item dynamika wpisów na Twitterze -- w związku z narzuconym w serwisie
  ograniczeniem na liczbę znaków wpisu (140) oraz sposobem ich prezentowania
  użytkownikom na ich stronie głównej (chronologicznie) serwis ten
  charakteryzuje się wysoką dynamiką informacji wysyłanych przez użytkowników.
  Wpisy bardzo często odnoszą się do aktualnych wydarzeń, krótko je komentując.
  W związku z tym badanie środowiska piłkarskiego ma duży sens, gdyż zainteresowani
  futbolem internauci mają wiele tematów do dyskusji -- komentowanie spotkań na żywo,
  refleksje po meczach i dyskusje między spotkaniami.
  W profesjonalnych ligach piłkarskich mecze odbywają się co najmniej raz w tygodniu,
  a informacje o stanie kadrowym, kontuzjach i nadziejach przed meczem
  rozpalają kibiców swoich drużyn. W związku z wysoką dynamiką świata piłkarskiego,
  użytkownicy Twittera będący kibicami tworzą proporcjonalnie dużą liczbę
  tweetów -- komentując na biężąco wszelkie wydarzenia. Dlatego też wybór
  tej podgrupy do moich badań jest racjonalny i daje możliwość zbadania 
  zachowań użytkowników społecznościowych dostarczając wiele tweetów, na których
  można przeprowadzić analizy.
\end{itemize}  


\subsection{Sposób zbierania danych}
% wybrane kluby, słowa kluczowe, API, nasłuchiwanie

\subsection{Podstawowe statystyki danych}


\clearpage
%%%%%%%%%%%%%%%%%%%%%%%%%%%%%%%%%%%%%%%%%%%%%%%%%%%%%%%%%%%% ANALIZA SENTYMENTU
\section{Analiza sentymentu}
\label{section:analizasentymentu}

\subsection{Normalizacja tekstu}
% pozbycie się słów kluczowych, zaprzeczenia, retweety
\subsection{Algorytm i jego modyfikacja}
% + dobór parametrów
\subsection{Aplikacja analizy sentymentu}
% przykładowe zdania i ich sentyment



%%%%%%%%%%%%%%%%%%%%%%%%%%%%%%%%%%%%%%%%%%%%%%%%%%%%% ANALIZA SIECI SPOŁECZNYCH
\section{Analiza sieci społecznych}
\label{section:siecispoleczne}
\subsection{Model grafowy}
% wykorzystanie reply, retweet
\subsection{Wykrywanie grup}
% gephi, modularity
\subsection{Badanie podobieństwa}
% algorytm podobieństwa grafów, podobieństwa krawędzi




%%%%%%%%%%%%%%%%%%%%%%%%%%%%%%%%%%%%%%%%%%%%%%%%%%%%%% WYKORZYSTANIE GEOLOKACJI
\section{Wykorzystanie geolokacji}
\label{section:wykorzystaniegeolokacji}
% tweety z geolokacją
% wyciąganie informacji o miejscu - Open Street Map
% zastosowanie: odl. między użytkownikiami, od stadionu, dzielnice